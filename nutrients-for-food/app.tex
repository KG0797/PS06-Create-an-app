% Options for packages loaded elsewhere
\PassOptionsToPackage{unicode}{hyperref}
\PassOptionsToPackage{hyphens}{url}
%
\documentclass[
]{article}
\usepackage{amsmath,amssymb}
\usepackage{lmodern}
\usepackage{iftex}
\ifPDFTeX
  \usepackage[T1]{fontenc}
  \usepackage[utf8]{inputenc}
  \usepackage{textcomp} % provide euro and other symbols
\else % if luatex or xetex
  \usepackage{unicode-math}
  \defaultfontfeatures{Scale=MatchLowercase}
  \defaultfontfeatures[\rmfamily]{Ligatures=TeX,Scale=1}
\fi
% Use upquote if available, for straight quotes in verbatim environments
\IfFileExists{upquote.sty}{\usepackage{upquote}}{}
\IfFileExists{microtype.sty}{% use microtype if available
  \usepackage[]{microtype}
  \UseMicrotypeSet[protrusion]{basicmath} % disable protrusion for tt fonts
}{}
\makeatletter
\@ifundefined{KOMAClassName}{% if non-KOMA class
  \IfFileExists{parskip.sty}{%
    \usepackage{parskip}
  }{% else
    \setlength{\parindent}{0pt}
    \setlength{\parskip}{6pt plus 2pt minus 1pt}}
}{% if KOMA class
  \KOMAoptions{parskip=half}}
\makeatother
\usepackage{xcolor}
\usepackage[margin=1in]{geometry}
\usepackage{color}
\usepackage{fancyvrb}
\newcommand{\VerbBar}{|}
\newcommand{\VERB}{\Verb[commandchars=\\\{\}]}
\DefineVerbatimEnvironment{Highlighting}{Verbatim}{commandchars=\\\{\}}
% Add ',fontsize=\small' for more characters per line
\usepackage{framed}
\definecolor{shadecolor}{RGB}{248,248,248}
\newenvironment{Shaded}{\begin{snugshade}}{\end{snugshade}}
\newcommand{\AlertTok}[1]{\textcolor[rgb]{0.94,0.16,0.16}{#1}}
\newcommand{\AnnotationTok}[1]{\textcolor[rgb]{0.56,0.35,0.01}{\textbf{\textit{#1}}}}
\newcommand{\AttributeTok}[1]{\textcolor[rgb]{0.77,0.63,0.00}{#1}}
\newcommand{\BaseNTok}[1]{\textcolor[rgb]{0.00,0.00,0.81}{#1}}
\newcommand{\BuiltInTok}[1]{#1}
\newcommand{\CharTok}[1]{\textcolor[rgb]{0.31,0.60,0.02}{#1}}
\newcommand{\CommentTok}[1]{\textcolor[rgb]{0.56,0.35,0.01}{\textit{#1}}}
\newcommand{\CommentVarTok}[1]{\textcolor[rgb]{0.56,0.35,0.01}{\textbf{\textit{#1}}}}
\newcommand{\ConstantTok}[1]{\textcolor[rgb]{0.00,0.00,0.00}{#1}}
\newcommand{\ControlFlowTok}[1]{\textcolor[rgb]{0.13,0.29,0.53}{\textbf{#1}}}
\newcommand{\DataTypeTok}[1]{\textcolor[rgb]{0.13,0.29,0.53}{#1}}
\newcommand{\DecValTok}[1]{\textcolor[rgb]{0.00,0.00,0.81}{#1}}
\newcommand{\DocumentationTok}[1]{\textcolor[rgb]{0.56,0.35,0.01}{\textbf{\textit{#1}}}}
\newcommand{\ErrorTok}[1]{\textcolor[rgb]{0.64,0.00,0.00}{\textbf{#1}}}
\newcommand{\ExtensionTok}[1]{#1}
\newcommand{\FloatTok}[1]{\textcolor[rgb]{0.00,0.00,0.81}{#1}}
\newcommand{\FunctionTok}[1]{\textcolor[rgb]{0.00,0.00,0.00}{#1}}
\newcommand{\ImportTok}[1]{#1}
\newcommand{\InformationTok}[1]{\textcolor[rgb]{0.56,0.35,0.01}{\textbf{\textit{#1}}}}
\newcommand{\KeywordTok}[1]{\textcolor[rgb]{0.13,0.29,0.53}{\textbf{#1}}}
\newcommand{\NormalTok}[1]{#1}
\newcommand{\OperatorTok}[1]{\textcolor[rgb]{0.81,0.36,0.00}{\textbf{#1}}}
\newcommand{\OtherTok}[1]{\textcolor[rgb]{0.56,0.35,0.01}{#1}}
\newcommand{\PreprocessorTok}[1]{\textcolor[rgb]{0.56,0.35,0.01}{\textit{#1}}}
\newcommand{\RegionMarkerTok}[1]{#1}
\newcommand{\SpecialCharTok}[1]{\textcolor[rgb]{0.00,0.00,0.00}{#1}}
\newcommand{\SpecialStringTok}[1]{\textcolor[rgb]{0.31,0.60,0.02}{#1}}
\newcommand{\StringTok}[1]{\textcolor[rgb]{0.31,0.60,0.02}{#1}}
\newcommand{\VariableTok}[1]{\textcolor[rgb]{0.00,0.00,0.00}{#1}}
\newcommand{\VerbatimStringTok}[1]{\textcolor[rgb]{0.31,0.60,0.02}{#1}}
\newcommand{\WarningTok}[1]{\textcolor[rgb]{0.56,0.35,0.01}{\textbf{\textit{#1}}}}
\usepackage{graphicx}
\makeatletter
\def\maxwidth{\ifdim\Gin@nat@width>\linewidth\linewidth\else\Gin@nat@width\fi}
\def\maxheight{\ifdim\Gin@nat@height>\textheight\textheight\else\Gin@nat@height\fi}
\makeatother
% Scale images if necessary, so that they will not overflow the page
% margins by default, and it is still possible to overwrite the defaults
% using explicit options in \includegraphics[width, height, ...]{}
\setkeys{Gin}{width=\maxwidth,height=\maxheight,keepaspectratio}
% Set default figure placement to htbp
\makeatletter
\def\fps@figure{htbp}
\makeatother
\setlength{\emergencystretch}{3em} % prevent overfull lines
\providecommand{\tightlist}{%
  \setlength{\itemsep}{0pt}\setlength{\parskip}{0pt}}
\setcounter{secnumdepth}{-\maxdimen} % remove section numbering
\ifLuaTeX
  \usepackage{selnolig}  % disable illegal ligatures
\fi
\IfFileExists{bookmark.sty}{\usepackage{bookmark}}{\usepackage{hyperref}}
\IfFileExists{xurl.sty}{\usepackage{xurl}}{} % add URL line breaks if available
\urlstyle{same} % disable monospaced font for URLs
\hypersetup{
  pdftitle={app.R},
  pdfauthor={Katherine Guo},
  hidelinks,
  pdfcreator={LaTeX via pandoc}}

\title{app.R}
\author{Katherine Guo}
\date{2023-03-04}

\begin{document}
\maketitle

\begin{Shaded}
\begin{Highlighting}[]
\CommentTok{\# This is a Shiny web application. You can run the application by clicking}
\CommentTok{\# the \textquotesingle{}Run App\textquotesingle{} button above.}
\CommentTok{\#}
\CommentTok{\# Find out more about building applications with Shiny here:}
\CommentTok{\#}
\CommentTok{\#    http://shiny.rstudio.com/}
\CommentTok{\#}

\FunctionTok{library}\NormalTok{(shinyWidgets)}
\FunctionTok{library}\NormalTok{(shiny)}
\FunctionTok{library}\NormalTok{(rsconnect)}
\end{Highlighting}
\end{Shaded}

\begin{verbatim}
## 
## 载入程辑包:'rsconnect'
\end{verbatim}

\begin{verbatim}
## The following object is masked from 'package:shiny':
## 
##     serverInfo
\end{verbatim}

\begin{Shaded}
\begin{Highlighting}[]
\FunctionTok{library}\NormalTok{(dplyr)}
\end{Highlighting}
\end{Shaded}

\begin{verbatim}
## 
## 载入程辑包:'dplyr'
\end{verbatim}

\begin{verbatim}
## The following objects are masked from 'package:stats':
## 
##     filter, lag
\end{verbatim}

\begin{verbatim}
## The following objects are masked from 'package:base':
## 
##     intersect, setdiff, setequal, union
\end{verbatim}

\begin{Shaded}
\begin{Highlighting}[]
\FunctionTok{library}\NormalTok{(ggplot2)}
\FunctionTok{library}\NormalTok{(readr)}
\NormalTok{nutrients\_csvfile }\OtherTok{\textless{}{-}} \FunctionTok{read.csv}\NormalTok{(}\StringTok{"C:/Users/Katherine Guo/Desktop/info 201/PS06{-}Create{-}an{-}app/nutrients\_csvfile.csv"}\NormalTok{)}

\NormalTok{ui }\OtherTok{\textless{}{-}} \FunctionTok{fluidPage}\NormalTok{(}
  \FunctionTok{titlePanel}\NormalTok{(}\StringTok{"Nutritional Facts for most common foods"}\NormalTok{),}
  
  \FunctionTok{tabsetPanel}\NormalTok{(}\AttributeTok{type =} \StringTok{"tabs"}\NormalTok{,}
              \FunctionTok{tabPanel}\NormalTok{(}\StringTok{"About"}\NormalTok{, }\FunctionTok{verbatimTextOutput}\NormalTok{(}\StringTok{"About"}\NormalTok{),}
                       \FunctionTok{helpText}\NormalTok{(}\FunctionTok{h4}\NormalTok{(}\StringTok{"This app uses a dataset contains a csv file with"}\NormalTok{, }\FunctionTok{strong}\NormalTok{(}\StringTok{"more than 300 foods"}\NormalTok{), }\StringTok{"each with the}
\StringTok{                                amount of"}\NormalTok{, }\FunctionTok{strong}\NormalTok{(}\StringTok{"Calories, Fats, Proteins, Saturated fats, Carbohydrates, Fibers"}\NormalTok{), }\StringTok{"labelled for each food."}\NormalTok{)),}
                       \FunctionTok{helpText}\NormalTok{(}\FunctionTok{h4}\NormalTok{(}\StringTok{"The dataset contains"}\NormalTok{, }\FunctionTok{em}\NormalTok{(}\StringTok{"329"}\NormalTok{), }\StringTok{"unique food values and"}\NormalTok{, }\FunctionTok{em}\NormalTok{(}\StringTok{"10 variables."}\NormalTok{))),}
                       \FunctionTok{helpText}\NormalTok{(}\FunctionTok{strong}\NormalTok{(}\StringTok{"Below is a small amount of data : "}\NormalTok{, }\AttributeTok{style =} \StringTok{"color:blue"}\NormalTok{)),}
                       \FunctionTok{tableOutput}\NormalTok{(}\StringTok{"sample\_data"}\NormalTok{)}
\NormalTok{              ),}
              
              \FunctionTok{tabPanel}\NormalTok{(}\StringTok{"Plots"}\NormalTok{,}
                       \FunctionTok{helpText}\NormalTok{(}\FunctionTok{h4}\NormalTok{(}\StringTok{"You are able to see the each Select the aspect you want to }
\StringTok{                             explore more, total cases or active cases, etc."}\NormalTok{)),}
                       \FunctionTok{helpText}\NormalTok{(}\FunctionTok{h4}\NormalTok{(}\StringTok{"You will see barplots of each category\textquotesingle{}s food with its food name and its calories."}\NormalTok{)),}
                       \FunctionTok{helpText}\NormalTok{(}\FunctionTok{h4}\NormalTok{(}\FunctionTok{em}\NormalTok{(}\StringTok{"You can choose the calories range you want to see. This will help you decide how big calories food}
\StringTok{                       you want to know"}\NormalTok{, }\AttributeTok{style =} \StringTok{"color:green"}\NormalTok{))),}
                       \FunctionTok{sidebarLayout}\NormalTok{(}
                         \FunctionTok{sidebarPanel}\NormalTok{(}
                           \FunctionTok{selectInput}\NormalTok{(}\StringTok{"category"}\NormalTok{, }
                                              \AttributeTok{label =} \StringTok{"which category to plot: "}\NormalTok{, }
                                              \AttributeTok{choices =} \FunctionTok{unique}\NormalTok{(nutrients\_csvfile}\SpecialCharTok{$}\NormalTok{Category), }
                                              \AttributeTok{selected =} \FunctionTok{unique}\NormalTok{(nutrients\_csvfile}\SpecialCharTok{$}\NormalTok{Category)), }
                           
                           \FunctionTok{sliderInput}\NormalTok{(}\StringTok{"calories\_range"}\NormalTok{, }
                                       \StringTok{"What range of calories to plot: "}\NormalTok{, }
                                       \AttributeTok{min =} \DecValTok{0}\NormalTok{, }
                                       \AttributeTok{value =} \FunctionTok{c}\NormalTok{(}\DecValTok{0}\NormalTok{, }\DecValTok{600}\NormalTok{),}
                                       \AttributeTok{max =} \DecValTok{1000} 
\NormalTok{                                       ),}
                           \FunctionTok{pickerInput}\NormalTok{(}
                             \AttributeTok{inputId =} \StringTok{"palette"}\NormalTok{,}
                             \AttributeTok{label =} \StringTok{"Select a palette:"}\NormalTok{,}
                             \AttributeTok{choices =} \FunctionTok{c}\NormalTok{(}\StringTok{"Standard"}\NormalTok{, }\StringTok{"Set 2"}\NormalTok{),}
                             \AttributeTok{selected =} \StringTok{"Standard"}
\NormalTok{                           )}
\NormalTok{                         ),}
                         
                         \FunctionTok{mainPanel}\NormalTok{(}
                           \FunctionTok{plotOutput}\NormalTok{(}\StringTok{"calories\_plot"}\NormalTok{)}
\NormalTok{                         )}
\NormalTok{                       )}
\NormalTok{              ),}
              
              \FunctionTok{tabPanel}\NormalTok{(}\StringTok{"Tables"}\NormalTok{,}
                       \FunctionTok{helpText}\NormalTok{(}\FunctionTok{h4}\NormalTok{(}\StringTok{"The tables will show the"}\NormalTok{, }\FunctionTok{strong}\NormalTok{(}\StringTok{"top 10"}\NormalTok{), }\StringTok{"food for each nutrient.}
\StringTok{                                   The nutrients includes : protein, Fat, Sat.Fat, Fiber, and Carbs"}\NormalTok{)),}
                      \FunctionTok{sidebarPanel}\NormalTok{(}
                        \FunctionTok{selectInput}\NormalTok{(}\StringTok{"nutrient"}\NormalTok{, }
                                    \AttributeTok{label =} \StringTok{"Choose a nutrient you want to see: "}\NormalTok{,}
                                    \AttributeTok{choices =} \FunctionTok{c}\NormalTok{(}\StringTok{"Protein"}\NormalTok{, }\StringTok{"Fat"}\NormalTok{, }\StringTok{"Sat.Fat"}\NormalTok{, }\StringTok{"Fiber"}\NormalTok{, }\StringTok{"Carbs"}\NormalTok{))}
\NormalTok{                      ),}
                      \FunctionTok{mainPanel}\NormalTok{(}
                        \FunctionTok{tableOutput}\NormalTok{(}\StringTok{"nutrients\_table"}\NormalTok{)}
\NormalTok{                      )}
\NormalTok{                ),}
\NormalTok{  )}
\NormalTok{)}


\CommentTok{\# Define server logic required to draw a histogram}
\NormalTok{server }\OtherTok{\textless{}{-}} \ControlFlowTok{function}\NormalTok{(input, output) \{}
\NormalTok{  output}\SpecialCharTok{$}\NormalTok{calories\_plot }\OtherTok{\textless{}{-}} \FunctionTok{renderPlot}\NormalTok{(\{}
\NormalTok{    filtered\_data }\OtherTok{\textless{}{-}}\NormalTok{ nutrients\_csvfile }\SpecialCharTok{\%\textgreater{}\%}
      \FunctionTok{filter}\NormalTok{(Category }\SpecialCharTok{==}\NormalTok{ input}\SpecialCharTok{$}\NormalTok{category,}
\NormalTok{             Calories }\SpecialCharTok{\textgreater{}=}\NormalTok{ input}\SpecialCharTok{$}\NormalTok{calories\_range[}\DecValTok{1}\NormalTok{],}
\NormalTok{             Calories }\SpecialCharTok{\textless{}=}\NormalTok{ input}\SpecialCharTok{$}\NormalTok{calories\_range[}\DecValTok{2}\NormalTok{])}
    \ControlFlowTok{if}\NormalTok{(input}\SpecialCharTok{$}\NormalTok{palette }\SpecialCharTok{==} \StringTok{"Standard"}\NormalTok{) \{}
\NormalTok{      my\_color }\OtherTok{\textless{}{-}} \FunctionTok{c}\NormalTok{(}\StringTok{"blue"}\NormalTok{)}
\NormalTok{    \} }\ControlFlowTok{else}\NormalTok{ \{}
\NormalTok{      my\_color }\OtherTok{\textless{}{-}} \FunctionTok{c}\NormalTok{(}\StringTok{"red"}\NormalTok{)}
\NormalTok{    \}}
    
    
    \FunctionTok{ggplot}\NormalTok{(filtered\_data, }\FunctionTok{aes}\NormalTok{(}\AttributeTok{x =}\NormalTok{ Calories, }\AttributeTok{y =}\NormalTok{ Food, }\AttributeTok{fill =}\NormalTok{ Category)) }\SpecialCharTok{+}
      \FunctionTok{geom\_col}\NormalTok{(}\AttributeTok{position =} \StringTok{"dodge"}\NormalTok{) }\SpecialCharTok{+}
      \FunctionTok{scale\_fill\_manual}\NormalTok{(}\AttributeTok{values =}\NormalTok{ my\_color) }\SpecialCharTok{+}
      \FunctionTok{labs}\NormalTok{(}\AttributeTok{title =} \StringTok{"Calories for Different Categories"}\NormalTok{,}
           \AttributeTok{x =} \StringTok{"Calories"}\NormalTok{, }\AttributeTok{y =} \StringTok{"Food"}\NormalTok{) }\SpecialCharTok{+}
      \FunctionTok{theme}\NormalTok{(}\AttributeTok{axis.text.x =} \FunctionTok{element\_text}\NormalTok{(}\AttributeTok{angle =} \DecValTok{90}\NormalTok{, }\AttributeTok{size =} \DecValTok{9}\NormalTok{))}
\NormalTok{  \})}
  
\NormalTok{  output}\SpecialCharTok{$}\NormalTok{nutrients\_table }\OtherTok{\textless{}{-}} \FunctionTok{renderTable}\NormalTok{(\{}
\NormalTok{    nutrients\_csvfile }\SpecialCharTok{\%\textgreater{}\%}
      \FunctionTok{select}\NormalTok{(Food, }\SpecialCharTok{!!}\FunctionTok{sym}\NormalTok{(input}\SpecialCharTok{$}\NormalTok{nutrient)) }\SpecialCharTok{\%\textgreater{}\%}
      \FunctionTok{filter}\NormalTok{(}\SpecialCharTok{!!}\FunctionTok{sym}\NormalTok{(input}\SpecialCharTok{$}\NormalTok{nutrient) }\SpecialCharTok{!=} \StringTok{"t"}\NormalTok{) }\SpecialCharTok{\%\textgreater{}\%}
      \FunctionTok{arrange}\NormalTok{(}\SpecialCharTok{!!}\FunctionTok{sym}\NormalTok{(input}\SpecialCharTok{$}\NormalTok{nutrient) }\SpecialCharTok{\%\textgreater{}\%} \FunctionTok{desc}\NormalTok{(), Food) }\SpecialCharTok{\%\textgreater{}\%}
      \FunctionTok{mutate}\NormalTok{(}\AttributeTok{Rank =} \FunctionTok{row\_number}\NormalTok{()) }\SpecialCharTok{\%\textgreater{}\%}
      \FunctionTok{top\_n}\NormalTok{(}\DecValTok{10}\NormalTok{)}\SpecialCharTok{\%\textgreater{}\%}
      \FunctionTok{select}\NormalTok{(Rank, Food, }\SpecialCharTok{!!}\FunctionTok{sym}\NormalTok{(input}\SpecialCharTok{$}\NormalTok{nutrient))}
\NormalTok{  \})}
  
\NormalTok{  output}\SpecialCharTok{$}\NormalTok{sample\_data }\OtherTok{\textless{}{-}} \FunctionTok{renderTable}\NormalTok{(\{}
\NormalTok{    nutrients\_csvfile }\SpecialCharTok{\%\textgreater{}\%} 
      \FunctionTok{select}\NormalTok{(Food, Measure,Grams,Calories,Protein,Fat, Sat.Fat,Fiber, Carbs, Category) }\SpecialCharTok{\%\textgreater{}\%} 
      \FunctionTok{head}\NormalTok{(}\DecValTok{10}\NormalTok{)}
\NormalTok{  \})}
  
 
\NormalTok{\}}

\CommentTok{\# Run the application }
\FunctionTok{shinyApp}\NormalTok{(}\AttributeTok{ui =}\NormalTok{ ui, }\AttributeTok{server =}\NormalTok{ server)}
\end{Highlighting}
\end{Shaded}


\end{document}
